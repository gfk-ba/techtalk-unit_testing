\begin{frame}
	\frametitle{What is a unit?}
	\begin{block}{Unit}<+->
		A unit is a collection of functions and variables that is \alert{self-contained}.
	\end{block}

	\begin{itemize}[<+-| highlight@+>]
		\item A unit encompasses \alert{all functionality} working on a certain piece of state, \alert{and} \alert{all state} needed by that functionality. (I.e.: it is self-contained.)
		\item So, no dependency on outside state!
		\item A unit needs proper design to eliminate direct dependencies and allow for proper unit testing.
		\begin{itemize}[<+-| highlight@+>]
			\item For example, use the SOLID design principles as guidelines.
		\end{itemize}
	\end{itemize}
\end{frame}


\begin{frame}
	\only<+->{Candidate units:}
	\begin{itemize}[<+-| highlight@+>]
		\item Functions
		\item Classes
		\item Namespaces
		\item Modules
	\end{itemize}
\end{frame}


\begin{frame}
	\frametitle{Example}
	\begin{block}{Find the unit to test...}<+->
		\lstinputlisting[language=Javascript]{section/examples/unit1.js}
	\end{block}
\end{frame}


\begin{frame}
	\frametitle{Example}
	\begin{block}{Refactored to support unit testing...}<+->
		\lstinputlisting[language=Javascript]{section/examples/unit2.js}
	\end{block}
\end{frame}


